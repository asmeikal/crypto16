\documentclass{article}

\usepackage{amsmath}

\renewcommand{\theenumi}{(\alph{enumi})}
\renewcommand{\theenumii}{\roman{enumii}}

\newcommand{\xor}{\oplus}
\renewcommand{\Pr}[1]{\ensuremath{\mathrm{Pr} \left[ {#1} \right]}}

\title{Homework 1}
\author{Michele Laurenti - 1603064}

\begin{document}

\maketitle

\section{Perfect Secrecy and One-Time Pad}

\begin{enumerate}
	\item The encryption scheme would not be perfectly secure, since $\Pr{M=m | C=m} = 0$ for any $m$, in contradiction with the definition of perfect secrecy.

	\item Perfect secrecy means that for any message $m$ $\Pr{M=m} = \Pr{M=m|C=c}$.
		Thus, if the distribution $M$ is such that for some $m, m'$ $\Pr{M=m} \neq \Pr{M=m'}$, we'd have that $\Pr{M=m|C=c} \neq \Pr{M=m'|C=c}$.

	\item Since the adversary is given $m$ and $\phi$ in this setting, the adversary could access $k = m \xor \phi$, and thus compute, for any $m'$, $\phi' = m \xor k$.
\end{enumerate}

\section{Universal Hashing}

\begin{enumerate}
	\item prova

	\item prova
\end{enumerate}

\section{One-Way Functions}

\begin{enumerate}
	\item prova

	\item prova
		\begin{enumerate}
			\item prova
			\item prova
			\item prova
			\item prova
		\end{enumerate}
\end{enumerate}

\section{Pseudorandom Generators}

\begin{enumerate}
	\item prova

	\item prova
		\begin{enumerate}
			\item prova
			\item prova
		\end{enumerate}
\end{enumerate}

\section{Pseudorandom Functions}

\begin{enumerate}
	\item prova
	\item prova
	\item prova
\end{enumerate}

\section{Secret-Key Encryption}

Prova.

\end{document}

