
\chapter{Introduction}

\chapterprecishere{Solomon,\\ I'm concerned about security; I think, when we email each other, we should use some sort of code.}

Confidentiality is our goal.
We want to encrypt and decrypt a (plaintext) message $m$, using a key, to obtain a cyphertext $c$.
As per Kirkoff's principle, only the key is secret.

% Drawing of message exchange: A, B, E

Our encryption schemes have the following syntax:
\begin{equation*}
	\GenericEncSchemeTuple.
\end{equation*}
$A$ and $B$, the actors of our communication exchange, share $k$, the key, taken from some key space $\K$.
The elements of our encryption scheme play the following roles:
\begin{enumerate}
	\item $\GenericKeyGen$ outputs a random key from the key space $\K$, and we write this as $k \rand{\GenericKeyGen}$;
	\item $\GenericEnc : \K \times \M \to \C$ is the encryption function, mapping a key and a message to a cyphertext;
	\item $\GenericDec : \K \times \C \to \M$ is the decryption function, mapping a key and a cyphertext to a message.
\end{enumerate}
We expect an encryption scheme to be at least correct:
\begin{equation*}
	\forall k \in \K, \forall m \in \M . \GenericDec(k, \GenericEnc(k, m)) = m.
\end{equation*}

\section{Perfect secrecy}

Shannon defined ``perfect secrecy'', \ie the fact that the cyphertext carries no information about the plaintext.
\begin{definition}[Perfect secrecy]\label{def:perfect-secrecy}
	Let $M$ be a \ac{RV} over $\M$, and $K$ be a uniform distribution over $\K$.

	$(\GenericEnc, \GenericDec)$ has perfect secrecy if
	\begin{equation*}
		\forall M, \forall m \in \M, c \in \C . \Pr{M = m} = \Pr{M = m | C = c}
	\end{equation*}
	where $C = \GenericEnc(k,m)$ is a third \ac{RV}.
\end{definition}

We have equivalent definitions for perfect secrecy.
\begin{theorem}\label{thm:perfect-secrecy:equivalent-definitions}
	The following definitions are equivalent:
	\begin{enumerate}
		\item \label{itm:thm:perfect-secrecy:original} \cref{def:perfect-secrecy};
		\item \label{itm:thm:perfect-secrecy:independent} $M$ and $C$ are independent;
		\item \label{itm:thm:perfect-secrecy:invariant} $\forall m, m' \in \M, \forall c \in \C$
			\begin{equation*}
				\Pr{\GenericEnc(k,m) = c} = \Pr{\GenericEnc(k,m') = c}
			\end{equation*}
			where $k$ is a random key in $\K$ chosen with uniform probability. \qedhere
	\end{enumerate}
\end{theorem}

\begin{proof}[Proof of \cref{thm:perfect-secrecy:equivalent-definitions}]
	First, we show that \ref{itm:thm:perfect-secrecy:original} implies \ref{itm:thm:perfect-secrecy:independent}.
	\begin{align*}
		\Pr{M=m} & = \Pr{M = m | C = c} \\
		& = \frac{\Pr{M = m \land C = c}}{\Pr{C = c}} \tag{by Bayes}
		\\
		& \implies \\
		\Pr{M = m} \Pr{C = c}
		& =
		\Pr{M = m \land C = c}
	\end{align*}
	which is the definition of independence.

	Now we show that \ref{itm:thm:perfect-secrecy:independent} implies \ref{itm:thm:perfect-secrecy:invariant}.
	Fix $m \in \M$ and $c \in \C$.
	\begin{align*}
		\Pr{\GenericEnc(k, m) = c} 
		& =
		\Pr{\GenericEnc(k, M) = c | M = m} \tag{we fixed $m$}
		\\
		& = \Pr{C = c | M = m} \tag{definition of the \ac{RV} $C$}
		\\
		& = \Pr{C = c}. \tag{by \ref{itm:thm:perfect-secrecy:independent}}
	\end{align*}
	Since $m$ is arbitrary, we can do the same for $m'$, and obtain
	\begin{equation*}
		\Pr{\GenericEnc(k,m') = c} = \Pr{C = c}
	\end{equation*}
	which gives us \ref{itm:thm:perfect-secrecy:invariant}.

	Now we want to show that \ref{itm:thm:perfect-secrecy:invariant} implies \ref{itm:thm:perfect-secrecy:original}.
	Take any $c \in \C$.
	\begin{align*}
		\Pr{C = c}
		& =
		\sum_{m' \in \M} \Pr{C = c \land M = m'}
		\\
		& = 
		\sum_{m' \in \M} \Pr{C = c | M = m'} \Pr{M = m'}
		\tag{by Bayes}
		\\
		& =
		\sum_{m' \in \M} \Pr{\GenericEnc(k,M) = c | M = m'} \Pr{M = m'}
		\\
		& =
		\sum_{m' \in \M} \Pr{\GenericEnc(k,m') = c} \Pr{M = m'}
		\\
		& =
		\Pr{\GenericEnc(k, m) = c} \underbrace{\sum_{m' \in \M} \Pr{M = m'}}_{1}
		\tag{$\GenericEnc$ is indepenendent of $M$, so we take it out}
		\\
		& =
		\Pr{\GenericEnc(k,M) = c | M = m} =
		\Pr{C = c | M = m}.
	\end{align*}
	We are left to show that $\Pr{M = m} = \Pr{M = m | C = c}$, but this is easy with Bayes.
\end{proof}

\subsection{\acl{OTP}}

Now we'll see a perfect encryption scheme, the \ac{OTP}.
The message space, the cyphertext space, and the key space are all the same, \ie $\M = \K = \C = \{0,1\}^{l}$, with $l \in \Naturals$.

Encryption and decryption use the xor operation:
\begin{itemize}
	\item $\OTPEnc(k,m) = k \xor m = c$;
	\item $\OTPDec(k,c) = c \xor k = (k \xor m) \xor k = m$.
\end{itemize}
Seeing that this is correct is immediate.

This can actually be done in any finite abelian group $(\Group, +)$, where you just do $k + m$ to encode and $c - k$ to decode.

\begin{theorem} \label{thm:otp:perfect-secrecy}
	\ac{OTP} is perfectly secure.
\end{theorem}

\begin{proof}[Proof of \cref{thm:otp:perfect-secrecy}]
	Fix $m \in \M, c \in \C$, and choose a random key.
	\begin{equation*}
		\Pr{\OTPEnc(k,m) = c} = \Pr{k = c - m} = \frac{1}{\abs{\Group}}.
	\end{equation*}
	This is true for any $m$, so we are done.
\end{proof}

\ac{OTP} has two problems:
\begin{enumerate}
	\item the key is long (as long as the message);
	\item we can't reuse the key:
	\begin{equation*}
		\begin{array}{c}
			c = k + m \\
			c' = k + m'
		\end{array}
		\implies
		c - c' = m - m'
		\implies
		m' = m - (c - c').
	\end{equation*}
\end{enumerate}

\begin{theorem}[Shannon, 1949] \label{thm:shannon:1949}
	In any perfectly secure encryption scheme the size of the key space is at least as large as the size of the message space, \ie $\abs{\K} \ge \abs{\M}$.
\end{theorem}

\begin{proof}[Proof of \cref{thm:shannon:1949}]
	Assume, for the sake of contradiction, that $\abs{\K} < \abs{\M}$.
	Fix $M$ to be the uniform distribution over $\M$, which we can do as perfect secrecy works for any distribution.
	Take a cyphertext $c \in \C$ such that $\Pr{C = c} > 0$, \ie $\exists m, k$ such that $\OTPEnc(k,m) = c$.

	Consider $\M' = \{ \OTPDec(k,c) : k \in \K \}$, the set of all messages decrypted from $c$ using any key.
	Clearly, $\abs{\M'} \le \abs{\K} < \abs{\M}$, so $\exists m' \in \M$ such that $m' \not\in \M'$.
	This means that
	\begin{equation*}
		\Pr{M = m'} = \frac{1}{\abs{\M}} \neq \Pr{M = m' | C = c} = 0
	\end{equation*}
	in contradiction with perfect secrecy.
\end{proof}
